\chapter*{Acknowledgements}
\thispagestyle{empty}

Questo lavoro segna una tappa importante di un percorso iniziato già qualche anno fa quando, implementando A* vidi per la prima volta un software che portava a termine un compito senza richiedere una esplicita soluzione da programmare. A* aveva però molti limiti, uno in particolare era la richiesta di un umano che si specializzasse abbastanza da riuscire a trovare un'euristica per quel problema.  Da quel momento nacque in me la volontà di creare un agente che fosse il più autonomo possibile, limitando in maniere sempre maggiore l'ingerenza umana. Ben presto iniziai a concentrare le mie energie sull'apprendimento per rinforzo, per permettere alla IA di imparare direttamente dalla propria esperienza, e sulle reti neurali per aumentare la potenza espressiva della mente dell'agente (percezione visiva, memoria, capacità rappresentativa).
\\
\newline
Quando ho iniziato questo percorso ero convinto che avrei saltato questa pagina, visto che ero circondato da gente scettica che non mi era di aiuto e anzi frenava il mio entusiasmo. Dopo poco fortunatamente trovai una via di fuga attraverso il web. Per cui il mio primo ringraziamento va a chiunque utilizzi questo potente mezzo per trasmettere conoscenza. In particolar modo vorrei ringraziare: Salman Khan, Andrew Ng, David Silver, Sergey Levine.
\\
\newline
Dopo un primo anno di studio autonomo e parallelo rispetto a quello universitario, avevo maturato una piccola conscenza di base che ho poi avuto modo di ampliare durante il mio percorso in Addfor. Il mio secondo ringraziamento va quindi a loro per avermi accolto e per aver creduto ed investito in me, fornendomi l'attrezzatura tecnica necessaria ed il supporto, sia sul piano tecnico che su quello umano. Un ringraziamento particolare ad Antonio, Sonia ed Enrico.
\\
\newline
Infine, il mio terzo ringraziamento va alla mia famiglia per il supporto, ai miei colleghi per avermi accompagnato in questo percorso ed ai miei amici (lontani e vicini) per avermi incoraggiato quando le aspettative si facevano vertiginose, per aver sopportato la mia assenza quando le scadenze si avvicinavano e per avermi sostenuto e consigliato quando gli esperimenti fallivano. 
Un ringraziamento particolare va a Erica per essere riuscita a farmi ritrovare la serenità persa in questi ultimi anni.

\clearpage